\PassOptionsToPackage{dvipsnames,table}{xcolor}
\documentclass[10pt]{beamer}
\usepackage{Cours}

\begin{document}

%QCM de NSI \QNSI{Question}{R1}{R2}{R3}{R4}
\newcommand{\QNSI}[5]{
#1
\begin{enumerate}
\item #2
\item #3
\item #4
\item #5
\end{enumerate}
}


\newcounter{numchap}
\setcounter{numchap}{1}
\newcounter{numframe}
\setcounter{numframe}{0}
\newcommand{\mframe}[1]{\frametitle{#1} \addtocounter{numframe}{1}}
\newcommand{\cnum}{\fbox{\textcolor{yellow}{\textbf{C\thenumchap}}}~}

\definecolor{grispale}{gray}{0.95}


\definecolor{grispale}{gray}{0.95}
\newcommand{\htmlmode}{\lstset{language=html,numbers=left, tabsize=2, frame=single, breaklines=true, keywordstyle=\ttfamily, basicstyle=\small,
   numberstyle=\tiny\ttfamily, framexleftmargin=0mm, backgroundcolor=\color{grispale}, xleftmargin=12mm,showstringspaces=false}}
\newcommand{\pythonmode}{\lstset{language=python,numbers=left, tabsize=2, frame=single, breaklines=true, keywordstyle=\ttfamily, basicstyle=\small,
   numberstyle=\tiny\ttfamily, framexleftmargin=0mm, backgroundcolor=\color{grispale}, xleftmargin=12mm, showstringspaces=false}}
\newcommand{\bashmode}{\lstset{language=bash,numbers=left, tabsize=2, frame=single, breaklines=true, basicstyle=\ttfamily,
   numberstyle=\tiny\ttfamily, framexleftmargin=0mm, backgroundcolor=\color{grispale}, xleftmargin=12mm, showstringspaces=false}}
\newcommand{\exomode}{\lstset{language=python,numbers=left, tabsize=2, frame=single, breaklines=true, basicstyle=\ttfamily,
   numberstyle=\tiny\ttfamily, framexleftmargin=13mm, xleftmargin=12mm, basicstyle=\small, showstringspaces=false}}
   
   
  \lstset{%
        inputencoding=utf8,
        extendedchars=true,
        literate=%
        {é}{{\'{e}}}1
        {è}{{\`{e}}}1
        {ê}{{\^{e}}}1
        {ë}{{\¨{e}}}1
        {É}{{\'{E}}}1
        {Ê}{{\^{E}}}1
        {û}{{\^{u}}}1
        {ù}{{\`{u}}}1
        {ú}{{\'{u}}}1
        {â}{{\^{a}}}1
        {à}{{\`{a}}}1
        {á}{{\'{a}}}1
        {ã}{{\~{a}}}1
        {Á}{{\'{A}}}1
        {Â}{{\^{A}}}1
        {Ã}{{\~{A}}}1
        {ç}{{\c{c}}}1
        {Ç}{{\c{C}}}1
        {õ}{{\~{o}}}1
        {ó}{{\'{o}}}1
        {ô}{{\^{o}}}1
        {Õ}{{\~{O}}}1
        {Ó}{{\'{O}}}1
        {Ô}{{\^{O}}}1
        {î}{{\^{i}}}1
        {Î}{{\^{I}}}1
        {í}{{\'{i}}}1
        {Í}{{\~{Í}}}1
}

%tei pour placer les images
%tei{nom de l’image}{échelle de l’image}{sens}{texte a positionner}
%sens ="1" (droite) ou "2" (gauche)
\newlength{\ltxt}
\newcommand{\tei}[4]{
\setlength{\ltxt}{\linewidth}
\setbox0=\hbox{\includegraphics[scale=#2]{#1}}
\addtolength{\ltxt}{-\wd0}
\addtolength{\ltxt}{-10pt}
\ifthenelse{\equal{#3}{1}}{
\begin{minipage}{\wd0}
\includegraphics[scale=#2]{#1}
\end{minipage}
\hfill
\begin{minipage}{\ltxt}
#4
\end{minipage}
}{
\begin{minipage}{\ltxt}
#4
\end{minipage}
\hfill
\begin{minipage}{\wd0}
\includegraphics[scale=#2]{#1}
\end{minipage}
}
}

%Juxtaposition d'une image pspciture et de texte 
%#1: = code pstricks de l'image
%#2: largeur de l'image
%#3: hauteur de l'image
%#4: Texte à écrire
\newcommand{\ptp}[4]{
\setlength{\ltxt}{\linewidth}
\addtolength{\ltxt}{-#2 cm}
\addtolength{\ltxt}{-0.1 cm}
\begin{minipage}[b][#3 cm][t]{\ltxt}
#4
\end{minipage}\hfill
\begin{minipage}[b][#3 cm][c]{#2 cm}
#1
\end{minipage}\par
}



%Macros pour les graphiques
\psset{linewidth=0.5\pslinewidth,PointSymbol=x}
\setlength{\fboxrule}{0.5pt}
\newcounter{tempangle}

%Marque la longueur du segment d'extrémité  #1 et  #2 avec la valeur #3, #4 est la distance par rapport au segment (en %age de la valeur de celui ci) et #5 l'orientation du marquage : +90 ou -90
\newcommand{\afflong}[5]{
\pstRotation[RotAngle=#4,PointSymbol=none,PointName=none]{#1}{#2}[X] 
\pstHomO[PointSymbol=none,PointName=none,HomCoef=#5]{#1}{X}[Y]
\pstTranslation[PointSymbol=none,PointName=none]{#1}{#2}{Y}[Z]
 \ncline{|<->|,linewidth=0.25\pslinewidth}{Y}{Z} \ncput*[nrot=:U]{\footnotesize{#3}}
}
\newcommand{\afflongb}[3]{
\ncline{|<->|,linewidth=0}{#1}{#2} \naput*[nrot=:U]{\footnotesize{#3}}
}

%Construis le point #4 situé à #2 cm du point #1 avant un angle #3 par rapport à l'horizontale. #5 = liste de paramètre
\newcommand{\lsegment}[5]{\pstGeonode[PointSymbol=none,PointName=none](0,0){O'}(#2,0){I'} \pstTranslation[PointSymbol=none,PointName=none]{O'}{I'}{#1}[J'] \pstRotation[RotAngle=#3,PointSymbol=x,#5]{#1}{J'}[#4]}
\newcommand{\tsegment}[5]{\pstGeonode[PointSymbol=none,PointName=none](0,0){O'}(#2,0){I'} \pstTranslation[PointSymbol=none,PointName=none]{O'}{I'}{#1}[J'] \pstRotation[RotAngle=#3,PointSymbol=x,#5]{#1}{J'}[#4] \pstLineAB{#4}{#1}}

%Construis le point #4 situé à #3 cm du point #1 et faisant un angle de  90° avec la droite (#1,#2) #5 = liste de paramètre
\newcommand{\psegment}[5]{
\pstGeonode[PointSymbol=none,PointName=none](0,0){O'}(#3,0){I'}
 \pstTranslation[PointSymbol=none,PointName=none]{O'}{I'}{#1}[J']
 \pstInterLC[PointSymbol=none,PointName=none]{#1}{#2}{#1}{J'}{M1}{M2} \pstRotation[RotAngle=-90,PointSymbol=x,#5]{#1}{M1}[#4]
  }
  
%Construis le point #4 situé à #3 cm du point #1 et faisant un angle de  #5° avec la droite (#1,#2) #6 = liste de paramètre
\newcommand{\mlogo}[6]{
\pstGeonode[PointSymbol=none,PointName=none](0,0){O'}(#3,0){I'}
 \pstTranslation[PointSymbol=none,PointName=none]{O'}{I'}{#1}[J']
 \pstInterLC[PointSymbol=none,PointName=none]{#1}{#2}{#1}{J'}{M1}{M2} \pstRotation[RotAngle=#5,PointSymbol=x,#6]{#1}{M2}[#4]
  }

% Construis un triangle avec #1=liste des 3 sommets séparés par des virgules, #2=liste des 3 longueurs séparés par des virgules, #3 et #4 : paramètre d'affichage des 2e et 3 points et #5 : inclinaison par rapport à l'horizontale
%autre macro identique mais sans tracer les segments joignant les sommets
\noexpandarg
\newcommand{\Triangleccc}[5]{
\StrBefore{#1}{,}[\pointA]
\StrBetween[1,2]{#1}{,}{,}[\pointB]
\StrBehind[2]{#1}{,}[\pointC]
\StrBefore{#2}{,}[\coteA]
\StrBetween[1,2]{#2}{,}{,}[\coteB]
\StrBehind[2]{#2}{,}[\coteC]
\tsegment{\pointA}{\coteA}{#5}{\pointB}{#3} 
\lsegment{\pointA}{\coteB}{0}{Z1}{PointSymbol=none, PointName=none}
\lsegment{\pointB}{\coteC}{0}{Z2}{PointSymbol=none, PointName=none}
\pstInterCC{\pointA}{Z1}{\pointB}{Z2}{\pointC}{Z3} 
\pstLineAB{\pointA}{\pointC} \pstLineAB{\pointB}{\pointC}
\pstSymO[PointName=\pointC,#4]{C}{C}[C]
}
\noexpandarg
\newcommand{\TrianglecccP}[5]{
\StrBefore{#1}{,}[\pointA]
\StrBetween[1,2]{#1}{,}{,}[\pointB]
\StrBehind[2]{#1}{,}[\pointC]
\StrBefore{#2}{,}[\coteA]
\StrBetween[1,2]{#2}{,}{,}[\coteB]
\StrBehind[2]{#2}{,}[\coteC]
\tsegment{\pointA}{\coteA}{#5}{\pointB}{#3} 
\lsegment{\pointA}{\coteB}{0}{Z1}{PointSymbol=none, PointName=none}
\lsegment{\pointB}{\coteC}{0}{Z2}{PointSymbol=none, PointName=none}
\pstInterCC[PointNameB=none,PointSymbolB=none,#4]{\pointA}{Z1}{\pointB}{Z2}{\pointC}{Z1} 
}


% Construis un triangle avec #1=liste des 3 sommets séparés par des virgules, #2=liste formée de 2 longueurs et d'un angle séparés par des virgules, #3 et #4 : paramètre d'affichage des 2e et 3 points et #5 : inclinaison par rapport à l'horizontale
%autre macro identique mais sans tracer les segments joignant les sommets
\newcommand{\Trianglecca}[5]{
\StrBefore{#1}{,}[\pointA]
\StrBetween[1,2]{#1}{,}{,}[\pointB]
\StrBehind[2]{#1}{,}[\pointC]
\StrBefore{#2}{,}[\coteA]
\StrBetween[1,2]{#2}{,}{,}[\coteB]
\StrBehind[2]{#2}{,}[\angleA]
\tsegment{\pointA}{\coteA}{#5}{\pointB}{#3} 
\setcounter{tempangle}{#5}
\addtocounter{tempangle}{\angleA}
\tsegment{\pointA}{\coteB}{\thetempangle}{\pointC}{#4}
\pstLineAB{\pointB}{\pointC}
}
\newcommand{\TriangleccaP}[5]{
\StrBefore{#1}{,}[\pointA]
\StrBetween[1,2]{#1}{,}{,}[\pointB]
\StrBehind[2]{#1}{,}[\pointC]
\StrBefore{#2}{,}[\coteA]
\StrBetween[1,2]{#2}{,}{,}[\coteB]
\StrBehind[2]{#2}{,}[\angleA]
\lsegment{\pointA}{\coteA}{#5}{\pointB}{#3} 
\setcounter{tempangle}{#5}
\addtocounter{tempangle}{\angleA}
\lsegment{\pointA}{\coteB}{\thetempangle}{\pointC}{#4}
}

% Construis un triangle avec #1=liste des 3 sommets séparés par des virgules, #2=liste formée de 1 longueurs et de deux angle séparés par des virgules, #3 et #4 : paramètre d'affichage des 2e et 3 points et #5 : inclinaison par rapport à l'horizontale
%autre macro identique mais sans tracer les segments joignant les sommets
\newcommand{\Trianglecaa}[5]{
\StrBefore{#1}{,}[\pointA]
\StrBetween[1,2]{#1}{,}{,}[\pointB]
\StrBehind[2]{#1}{,}[\pointC]
\StrBefore{#2}{,}[\coteA]
\StrBetween[1,2]{#2}{,}{,}[\angleA]
\StrBehind[2]{#2}{,}[\angleB]
\tsegment{\pointA}{\coteA}{#5}{\pointB}{#3} 
\setcounter{tempangle}{#5}
\addtocounter{tempangle}{\angleA}
\lsegment{\pointA}{1}{\thetempangle}{Z1}{PointSymbol=none, PointName=none}
\setcounter{tempangle}{#5}
\addtocounter{tempangle}{180}
\addtocounter{tempangle}{-\angleB}
\lsegment{\pointB}{1}{\thetempangle}{Z2}{PointSymbol=none, PointName=none}
\pstInterLL[#4]{\pointA}{Z1}{\pointB}{Z2}{\pointC}
\pstLineAB{\pointA}{\pointC}
\pstLineAB{\pointB}{\pointC}
}
\newcommand{\TrianglecaaP}[5]{
\StrBefore{#1}{,}[\pointA]
\StrBetween[1,2]{#1}{,}{,}[\pointB]
\StrBehind[2]{#1}{,}[\pointC]
\StrBefore{#2}{,}[\coteA]
\StrBetween[1,2]{#2}{,}{,}[\angleA]
\StrBehind[2]{#2}{,}[\angleB]
\lsegment{\pointA}{\coteA}{#5}{\pointB}{#3} 
\setcounter{tempangle}{#5}
\addtocounter{tempangle}{\angleA}
\lsegment{\pointA}{1}{\thetempangle}{Z1}{PointSymbol=none, PointName=none}
\setcounter{tempangle}{#5}
\addtocounter{tempangle}{180}
\addtocounter{tempangle}{-\angleB}
\lsegment{\pointB}{1}{\thetempangle}{Z2}{PointSymbol=none, PointName=none}
\pstInterLL[#4]{\pointA}{Z1}{\pointB}{Z2}{\pointC}
}

%Construction d'un cercle de centre #1 et de rayon #2 (en cm)
\newcommand{\Cercle}[2]{
\lsegment{#1}{#2}{0}{Z1}{PointSymbol=none, PointName=none}
\pstCircleOA{#1}{Z1}
}

%construction d'un parallélogramme #1 = liste des sommets, #2 = liste contenant les longueurs de 2 côtés consécutifs et leurs angles;  #3, #4 et #5 : paramètre d'affichage des sommets #6 inclinaison par rapport à l'horizontale 
% meme macro sans le tracé des segements
\newcommand{\Para}[6]{
\StrBefore{#1}{,}[\pointA]
\StrBetween[1,2]{#1}{,}{,}[\pointB]
\StrBetween[2,3]{#1}{,}{,}[\pointC]
\StrBehind[3]{#1}{,}[\pointD]
\StrBefore{#2}{,}[\longueur]
\StrBetween[1,2]{#2}{,}{,}[\largeur]
\StrBehind[2]{#2}{,}[\angle]
\tsegment{\pointA}{\longueur}{#6}{\pointB}{#3} 
\setcounter{tempangle}{#6}
\addtocounter{tempangle}{\angle}
\tsegment{\pointA}{\largeur}{\thetempangle}{\pointD}{#5}
\pstMiddleAB[PointName=none,PointSymbol=none]{\pointB}{\pointD}{Z1}
\pstSymO[#4]{Z1}{\pointA}[\pointC]
\pstLineAB{\pointB}{\pointC}
\pstLineAB{\pointC}{\pointD}
}
\newcommand{\ParaP}[6]{
\StrBefore{#1}{,}[\pointA]
\StrBetween[1,2]{#1}{,}{,}[\pointB]
\StrBetween[2,3]{#1}{,}{,}[\pointC]
\StrBehind[3]{#1}{,}[\pointD]
\StrBefore{#2}{,}[\longueur]
\StrBetween[1,2]{#2}{,}{,}[\largeur]
\StrBehind[2]{#2}{,}[\angle]
\lsegment{\pointA}{\longueur}{#6}{\pointB}{#3} 
\setcounter{tempangle}{#6}
\addtocounter{tempangle}{\angle}
\lsegment{\pointA}{\largeur}{\thetempangle}{\pointD}{#5}
\pstMiddleAB[PointName=none,PointSymbol=none]{\pointB}{\pointD}{Z1}
\pstSymO[#4]{Z1}{\pointA}[\pointC]
}


%construction d'un cerf-volant #1 = liste des sommets, #2 = liste contenant les longueurs de 2 côtés consécutifs et leurs angles;  #3, #4 et #5 : paramètre d'affichage des sommets #6 inclinaison par rapport à l'horizontale 
% meme macro sans le tracé des segements
\newcommand{\CerfVolant}[6]{
\StrBefore{#1}{,}[\pointA]
\StrBetween[1,2]{#1}{,}{,}[\pointB]
\StrBetween[2,3]{#1}{,}{,}[\pointC]
\StrBehind[3]{#1}{,}[\pointD]
\StrBefore{#2}{,}[\longueur]
\StrBetween[1,2]{#2}{,}{,}[\largeur]
\StrBehind[2]{#2}{,}[\angle]
\tsegment{\pointA}{\longueur}{#6}{\pointB}{#3} 
\setcounter{tempangle}{#6}
\addtocounter{tempangle}{\angle}
\tsegment{\pointA}{\largeur}{\thetempangle}{\pointD}{#5}
\pstOrtSym[#4]{\pointB}{\pointD}{\pointA}[\pointC]
\pstLineAB{\pointB}{\pointC}
\pstLineAB{\pointC}{\pointD}
}

%construction d'un quadrilatère quelconque #1 = liste des sommets, #2 = liste contenant les longueurs des 4 côtés et l'angle entre 2 cotés consécutifs  #3, #4 et #5 : paramètre d'affichage des sommets #6 inclinaison par rapport à l'horizontale 
% meme macro sans le tracé des segements
\newcommand{\Quadri}[6]{
\StrBefore{#1}{,}[\pointA]
\StrBetween[1,2]{#1}{,}{,}[\pointB]
\StrBetween[2,3]{#1}{,}{,}[\pointC]
\StrBehind[3]{#1}{,}[\pointD]
\StrBefore{#2}{,}[\coteA]
\StrBetween[1,2]{#2}{,}{,}[\coteB]
\StrBetween[2,3]{#2}{,}{,}[\coteC]
\StrBetween[3,4]{#2}{,}{,}[\coteD]
\StrBehind[4]{#2}{,}[\angle]
\tsegment{\pointA}{\coteA}{#6}{\pointB}{#3} 
\setcounter{tempangle}{#6}
\addtocounter{tempangle}{\angle}
\tsegment{\pointA}{\coteD}{\thetempangle}{\pointD}{#5}
\lsegment{\pointB}{\coteB}{0}{Z1}{PointSymbol=none, PointName=none}
\lsegment{\pointD}{\coteC}{0}{Z2}{PointSymbol=none, PointName=none}
\pstInterCC[PointNameA=none,PointSymbolA=none,#4]{\pointB}{Z1}{\pointD}{Z2}{Z3}{\pointC} 
\pstLineAB{\pointB}{\pointC}
\pstLineAB{\pointC}{\pointD}
}


% Définition des colonnes centrées ou à droite pour tabularx
\newcolumntype{Y}{>{\centering\arraybackslash}X}
\newcolumntype{Z}{>{\flushright\arraybackslash}X}

%Les pointillés à remplir par les élèves
\newcommand{\po}[1]{\makebox[#1 cm]{\dotfill}}
\newcommand{\lpo}[1][3]{%
\multido{}{#1}{\makebox[\linewidth]{\dotfill}
}}

%Liste des pictogrammes utilisés sur la fiche d'exercice ou d'activités
\newcommand{\bombe}{\faBomb}
\newcommand{\livre}{\faBook}
\newcommand{\calculatrice}{\faCalculator}
\newcommand{\oral}{\faCommentO}
\newcommand{\surfeuille}{\faEdit}
\newcommand{\ordinateur}{\faLaptop}
\newcommand{\ordi}{\faDesktop}
\newcommand{\ciseaux}{\faScissors}
\newcommand{\danger}{\faExclamationTriangle}
\newcommand{\out}{\faSignOut}
\newcommand{\cadeau}{\faGift}
\newcommand{\flash}{\faBolt}
\newcommand{\lumiere}{\faLightbulb}
\newcommand{\compas}{\dsmathematical}
\newcommand{\calcullitteral}{\faTimesCircleO}
\newcommand{\raisonnement}{\faCogs}
\newcommand{\recherche}{\faSearch}
\newcommand{\rappel}{\faHistory}
\newcommand{\video}{\faFilm}
\newcommand{\capacite}{\faPuzzlePiece}
\newcommand{\aide}{\faLifeRing}
\newcommand{\loin}{\faExternalLink}
\newcommand{\groupe}{\faUsers}
\newcommand{\bac}{\faGraduationCap}
\newcommand{\histoire}{\faUniversity}
\newcommand{\coeur}{\faSave}
\newcommand{\python}{\faPython}
\newcommand{\os}{\faMicrochip}
\newcommand{\rd}{\faCubes}
\newcommand{\data}{\faColumns}
\newcommand{\web}{\faCode}
\newcommand{\prog}{\faFile}
\newcommand{\algo}{\faCogs}
\newcommand{\important}{\faExclamationCircle}
\newcommand{\maths}{\faTimesCircle}
% Traitement des données en tables
\newcommand{\tables}{\faColumns}
% Types construits
\newcommand{\construits}{\faCubes}
% Type et valeurs de base
\newcommand{\debase}{{\footnotesize \faCube}}
% Systèmes d'exploitation
\newcommand{\linux}{\faLinux}
\newcommand{\sd}{\faProjectDiagram}
\newcommand{\bd}{\faDatabase}

%Les ensembles de nombres
\renewcommand{\N}{\mathbb{N}}
\newcommand{\D}{\mathbb{D}}
\newcommand{\Z}{\mathbb{Z}}
\newcommand{\Q}{\mathbb{Q}}
\newcommand{\R}{\mathbb{R}}
\newcommand{\C}{\mathbb{C}}

%Ecriture des vecteurs
\newcommand{\vect}[1]{\vbox{\halign{##\cr 
  \tiny\rightarrowfill\cr\noalign{\nointerlineskip\vskip1pt} 
  $#1\mskip2mu$\cr}}}


%Compteur activités/exos et question et mise en forme titre et questions
\newcounter{numact}
\setcounter{numact}{1}
\newcounter{numseance}
\setcounter{numseance}{1}
\newcounter{numexo}
\setcounter{numexo}{0}
\newcounter{numprojet}
\setcounter{numprojet}{0}
\newcounter{numquestion}
\newcommand{\espace}[1]{\rule[-1ex]{0pt}{#1 cm}}
\newcommand{\Quest}[3]{
\addtocounter{numquestion}{1}
\begin{tabularx}{\textwidth}{X|m{1cm}|}
\cline{2-2}
\textbf{\sffamily{\alph{numquestion})}} #1 & \dots / #2 \\
\hline 
\multicolumn{2}{|l|}{\espace{#3}} \\
\hline
\end{tabularx}
}
\newcommand{\QuestR}[3]{
\addtocounter{numquestion}{1}
\begin{tabularx}{\textwidth}{X|m{1cm}|}
\cline{2-2}
\textbf{\sffamily{\alph{numquestion})}} #1 & \dots / #2 \\
\hline 
\multicolumn{2}{|l|}{\cor{#3}} \\
\hline
\end{tabularx}
}
\newcommand{\Pre}{{\sc nsi} 1\textsuperscript{e}}
\newcommand{\Term}{{\sc nsi} Terminale}
\newcommand{\Sec}{2\textsuperscript{e}}
\newcommand{\Exo}[2]{ \addtocounter{numexo}{1} \ding{113} \textbf{\sffamily{Exercice \thenumexo}} : \textit{#1} \hfill #2  \setcounter{numquestion}{0}}
\newcommand{\Projet}[1]{ \addtocounter{numprojet}{1} \ding{118} \textbf{\sffamily{Projet \thenumprojet}} : \textit{#1}}
\newcommand{\ExoD}[2]{ \addtocounter{numexo}{1} \ding{113} \textbf{\sffamily{Exercice \thenumexo}}  \textit{(#1 pts)} \hfill #2  \setcounter{numquestion}{0}}
\newcommand{\ExoB}[2]{ \addtocounter{numexo}{1} \ding{113} \textbf{\sffamily{Exercice \thenumexo}}  \textit{(Bonus de +#1 pts maximum)} \hfill #2  \setcounter{numquestion}{0}}
\newcommand{\Act}[2]{ \ding{113} \textbf{\sffamily{Activité \thenumact}} : \textit{#1} \hfill #2  \addtocounter{numact}{1} \setcounter{numquestion}{0}}
\newcommand{\Seance}{ \rule{1.5cm}{0.5pt}\raisebox{-3pt}{\framebox[4cm]{\textbf{\sffamily{Séance \thenumseance}}}}\hrulefill  \\
  \addtocounter{numseance}{1}}
\newcommand{\Acti}[2]{ {\footnotesize \ding{117}} \textbf{\sffamily{Activité \thenumact}} : \textit{#1} \hfill #2  \addtocounter{numact}{1} \setcounter{numquestion}{0}}
\newcommand{\titre}[1]{\begin{Large}\textbf{\ding{118}}\end{Large} \begin{large}\textbf{ #1}\end{large} \vspace{0.2cm}}
\newcommand{\QListe}[1][0]{
\ifthenelse{#1=0}
{\begin{enumerate}[partopsep=0pt,topsep=0pt,parsep=0pt,itemsep=0pt,label=\textbf{\sffamily{\arabic*.}},series=question]}
{\begin{enumerate}[resume*=question]}}
\newcommand{\SQListe}[1][0]{
\ifthenelse{#1=0}
{\begin{enumerate}[partopsep=0pt,topsep=0pt,parsep=0pt,itemsep=0pt,label=\textbf{\sffamily{\alph*)}},series=squestion]}
{\begin{enumerate}[resume*=squestion]}}
\newcommand{\SQListeL}[1][0]{
\ifthenelse{#1=0}
{\begin{enumerate*}[partopsep=0pt,topsep=0pt,parsep=0pt,itemsep=0pt,label=\textbf{\sffamily{\alph*)}},series=squestion]}
{\begin{enumerate*}[resume*=squestion]}}
\newcommand{\FinListe}{\end{enumerate}}
\newcommand{\FinListeL}{\end{enumerate*}}

%Mise en forme de la correction
\newcommand{\cor}[1]{\par \textcolor{OliveGreen}{#1}}
\newcommand{\br}[1]{\cor{\textbf{#1}}}
\newcommand{\tcor}[1]{\begin{tcolorbox}[width=0.92\textwidth,colback={white},colbacktitle=white,coltitle=OliveGreen,colframe=green!75!black,boxrule=0.2mm]   
\cor{#1}
\end{tcolorbox}
}
\newcommand{\rc}[1]{\textcolor{OliveGreen}{#1}}

%Référence aux exercices par leur numéro
\newcommand{\refexo}[1]{
\refstepcounter{numexo}
\addtocounter{numexo}{-1}
\label{#1}}

%Séparation entre deux activités
\newcommand{\separateur}{\begin{center}
\rule{1.5cm}{0.5pt}\raisebox{-3pt}{\ding{117}}\rule{1.5cm}{0.5pt}  \vspace{0.2cm}
\end{center}}

%Entête et pied de page
\newcommand{\snt}[1]{\lhead{\textbf{SNT -- La photographie numérique} \rhead{\textit{Lycée Nord}}}}
\newcommand{\Activites}[2]{\lhead{\textbf{{\sc #1}}}
\rhead{Activités -- \textbf{#2}}
\cfoot{}}
\newcommand{\Exos}[2]{\lhead{\textbf{Fiche d'exercices: {\sc #1}}}
\rhead{Niveau: \textbf{#2}}
\cfoot{}}
\newcommand{\Devoir}[2]{\lhead{\textbf{Devoir de mathématiques : {\sc #1}}}
\rhead{\textbf{#2}} \setlength{\fboxsep}{8pt}
\begin{center}
%Titre de la fiche
\fbox{\parbox[b][1cm][t]{0.3\textwidth}{Nom : \hfill \po{3} \par \vfill Prénom : \hfill \po{3}} } \hfill 
\fbox{\parbox[b][1cm][t]{0.6\textwidth}{Note : \po{1} / 20} }
\end{center} \cfoot{}}

%Devoir programmation en NSI (pas à rendre sur papier)
\newcommand{\PNSI}[2]{\lhead{\textbf{Devoir de {\sc nsi} : \textsf{ #1}}}
\rhead{\textbf{#2}} \setlength{\fboxsep}{8pt}
\begin{tcolorbox}[title=\textcolor{black}{\danger\; A lire attentivement},colbacktitle=lightgray]
{\begin{enumerate}
\item Rendre tous vous programmes en les envoyant par mail à l'adresse {\tt fnativel2@ac-reunion.fr}, en précisant bien dans le sujet vos noms et prénoms
\item Un programme qui fonctionne mal ou pas du tout peut rapporter des points
\item Les bonnes pratiques de programmation (clarté et lisiblité du code) rapportent des points
\end{enumerate}
}
\end{tcolorbox}
 \cfoot{}}


%Devoir de NSI
\newcommand{\DNSI}[2]{\lhead{\textbf{Devoir de {\sc nsi} : \textsf{ #1}}}
\rhead{\textbf{#2}} \setlength{\fboxsep}{8pt}
\begin{center}
%Titre de la fiche
\fbox{\parbox[b][1cm][t]{0.3\textwidth}{Nom : \hfill \po{3} \par \vfill Prénom : \hfill \po{3}} } \hfill 
\fbox{\parbox[b][1cm][t]{0.6\textwidth}{Note : \po{1} / 10} }
\end{center} \cfoot{}}

\newcommand{\DevoirNSI}[2]{\lhead{\textbf{Devoir de {\sc nsi} : {\sc #1}}}
\rhead{\textbf{#2}} \setlength{\fboxsep}{8pt}
\cfoot{}}

%La définition de la commande QCM pour auto-multiple-choice
%En premier argument le sujet du qcm, deuxième argument : la classe, 3e : la durée prévue et #4 : présence ou non de questions avec plusieurs bonnes réponses
\newcommand{\QCM}[4]{
{\large \textbf{\ding{52} QCM : #1}} -- Durée : \textbf{#3 min} \hfill {\large Note : \dots/10} 
\hrule \vspace{0.1cm}\namefield{}
Nom :  \textbf{\textbf{\nom{}}} \qquad \qquad Prénom :  \textbf{\prenom{}}  \hfill Classe: \textbf{#2}
\vspace{0.2cm}
\hrule  
\begin{itemize}[itemsep=0pt]
\item[-] \textit{Une bonne réponse vaut un point, une absence de réponse n'enlève pas de point. }
\item[\danger] \textit{Une mauvaise réponse enlève un point.}
\ifthenelse{#4=1}{\item[-] \textit{Les questions marquées du symbole \multiSymbole{} peuvent avoir plusieurs bonnes réponses possibles.}}{}
\end{itemize}
}
\newcommand{\DevoirC}[2]{
\renewcommand{\footrulewidth}{0.5pt}
\lhead{\textbf{Devoir de mathématiques : {\sc #1}}}
\rhead{\textbf{#2}} \setlength{\fboxsep}{8pt}
\fbox{\parbox[b][0.4cm][t]{0.955\textwidth}{Nom : \po{5} \hfill Prénom : \po{5} \hfill Classe: \textbf{1}\textsuperscript{$\dots$}} } 
\rfoot{\thepage} \cfoot{} \lfoot{Lycée Nord}}
\newcommand{\DevoirInfo}[2]{\lhead{\textbf{Evaluation : {\sc #1}}}
\rhead{\textbf{#2}} \setlength{\fboxsep}{8pt}
 \cfoot{}}
\newcommand{\DM}[2]{\lhead{\textbf{Devoir maison à rendre le #1}} \rhead{\textbf{#2}}}

%Macros permettant l'affichage des touches de la calculatrice
%Touches classiques : #1 = 0 fond blanc pour les nombres et #1= 1gris pour les opérations et entrer, second paramètre=contenu
%Si #2=1 touche arrondi avec fond gris
\newcommand{\TCalc}[2]{
\setlength{\fboxsep}{0.1pt}
\ifthenelse{#1=0}
{\psframebox[fillstyle=solid, fillcolor=white]{\parbox[c][0.25cm][c]{0.6cm}{\centering #2}}}
{\ifthenelse{#1=1}
{\psframebox[fillstyle=solid, fillcolor=lightgray]{\parbox[c][0.25cm][c]{0.6cm}{\centering #2}}}
{\psframebox[framearc=.5,fillstyle=solid, fillcolor=white]{\parbox[c][0.25cm][c]{0.6cm}{\centering #2}}}
}}
\newcommand{\Talpha}{\psdblframebox[fillstyle=solid, fillcolor=white]{\hspace{-0.05cm}\parbox[c][0.25cm][c]{0.65cm}{\centering \scriptsize{alpha}}} \;}
\newcommand{\Tsec}{\psdblframebox[fillstyle=solid, fillcolor=white]{\parbox[c][0.25cm][c]{0.6cm}{\centering \scriptsize 2nde}} \;}
\newcommand{\Tfx}{\psdblframebox[fillstyle=solid, fillcolor=white]{\parbox[c][0.25cm][c]{0.6cm}{\centering \scriptsize $f(x)$}} \;}
\newcommand{\Tvar}{\psframebox[framearc=.5,fillstyle=solid, fillcolor=white]{\hspace{-0.22cm} \parbox[c][0.25cm][c]{0.82cm}{$\scriptscriptstyle{X,T,\theta,n}$}}}
\newcommand{\Tgraphe}{\psdblframebox[fillstyle=solid, fillcolor=white]{\hspace{-0.08cm}\parbox[c][0.25cm][c]{0.68cm}{\centering \tiny{graphe}}} \;}
\newcommand{\Tfen}{\psdblframebox[fillstyle=solid, fillcolor=white]{\hspace{-0.08cm}\parbox[c][0.25cm][c]{0.68cm}{\centering \tiny{fenêtre}}} \;}
\newcommand{\Ttrace}{\psdblframebox[fillstyle=solid, fillcolor=white]{\parbox[c][0.25cm][c]{0.6cm}{\centering \scriptsize{trace}}} \;}

% Macroi pour l'affichage  d'un entier n dans  une base b
\newcommand{\base}[2]{\left( #1 \right)_{#2}}}
\setcounter{numchap}{2}
\newcommand{\Encodage}{\cnum Représentation des entiers, encodage des caractères}
\pythonmode

% Ecriture des nombres avec des 0 et des 1
\begin{frame}
	\mframe{\Encodage}
	\begin{alertblock}{Ecriture binaire}
		\begin{itemize}
			\item<1-> On peut écrire les nombres entiers positifs en utilisant seulement deux chiffres : 0 et 1.
			\item<2-> Chaque chiffre est multiplié par une puissance de 2 selon  sa position dans le nombre.
		\end{itemize}
	\end{alertblock}
	\begin{exampleblock}{Exemple}
		\onslide<3->{Par exemple en binaire le nombre \textcolor{red}{$10001011$} correspond à 139 en décimal : \\}
		\onslide<4->{
			\begin{tabular}{p{0.4cm}p{0.4cm}p{0.4cm}p{0.4cm}p{0.4cm}p{0.4cm}p{0.4cm}p{0.4cm}c}
				                   &                    &                    &                    &                    &                    &                    &                    & \\
				\textcolor{red}{1} & \textcolor{red}{0} & \textcolor{red}{0} & \textcolor{red}{0} & \textcolor{red}{1} & \textcolor{red}{0} & \textcolor{red}{1} & \textcolor{red}{1}   \\
			\end{tabular}}
	\end{exampleblock}
\end{frame}


% Ecriture des nombres avec des 0 et des 1
\begin{frame}
	\mframe{\Encodage}
	\begin{alertblock}{Ecriture binaire}
		\begin{itemize}
			\item On peut écrire les nombres entiers positifs en utilisant seulement deux chiffres : 0 et 1.
			\item Chaque chiffre est multiplié par une puissance de 2 selon  sa position dans le nombre.
		\end{itemize}
	\end{alertblock}
	\begin{exampleblock}{Exemple}
		Par exemple en binaire le nombre \textcolor{red}{$10001011$} correspond à 139 en décimal : \\
		\begin{tabular}{|p{0.4cm}|p{0.4cm}|p{0.4cm}|p{0.4cm}|p{0.4cm}|p{0.4cm}|p{0.4cm}|p{0.4cm}|l}
			\textcolor{blue}{$\scriptstyle{2^7}$} & \textcolor{blue}{$\scriptstyle{2^6}$}   & \textcolor{blue}{$\scriptstyle{2^5}$} & \textcolor{blue}{$\scriptstyle{2^4}$} & \textcolor{blue}{$\scriptstyle{2^3}$} & \textcolor{blue}{$\scriptstyle{2^2}$} & \textcolor{blue}{$\scriptstyle{2^1}$} & \textcolor{blue}{$\scriptstyle{2^0}$}                                                                                                                                                                                                                                 \\
			\cline{1-8}
			\textcolor{red}{1}                    & \textcolor{red}{0}                      & \textcolor{red}{0}                    & \textcolor{red}{0}                    & \textcolor{red}{1}                    & \textcolor{red}{0}                    & \textcolor{red}{1}                    & \textcolor{red}{1}                    & = \onslide<2->{\textcolor{red}{1}$\times$ \textcolor{blue}{$2^7$}+\textcolor{red}{1}$\times$ \textcolor{blue}{$2^3$}+\textcolor{red}{1}$\times$ \textcolor{blue}{$2^1$} + \textcolor{red}{1}$\times$ \textcolor{blue}{$2^0$}} \\
			\multicolumn{8}{l}{}                  & \onslide<3->{$= 128 + 8 + 2 + 1 = 139$}                                                                                                                                                                                                                                                                                                                                                                                                                                                                                 \\
		\end{tabular}
	\end{exampleblock}
\end{frame}

% Remarques
\begin{frame}
	\mframe{\Encodage}
	\begin{block}{Remarque sur l'écriture décimale :}
		Nous sommes habitués à écrire les nombres en base 10, et en utilisant 10 chiffres (0,1,2,3,4,5,6,7,8 et 9), mais c'est le \textbf{même} principe qui est utilisé : les chiffres d'un nombre sont multipliés par une puissance de 10 suivant leur emplacement dans le nombre.\\
		\onslide<2->{Par exemple, pour \textcolor{red}{1815} :\\}
		\onslide<3->{\begin{tabular}{p{0.4cm}p{0.4cm}p{0.4cm}p{0.4cm}c}
				                   &                    &                    &                    & \\
				\textcolor{red}{1} & \textcolor{red}{8} & \textcolor{red}{1} & \textcolor{red}{5} & \\
			\end{tabular}
		}
	\end{block}
\end{frame}

\begin{frame}
	\mframe{\Encodage}
	\begin{block}{Remarque sur l'écriture décimale :}
		Nous sommes habitués à écrire les nombres en base 10, et en utilisant 10 chiffres (0,1,2,3,4,5,6,7,8 et 9), mais c'est le \textbf{même} principe qui est utilisé : les chiffres d'un nombre sont multipliés par une puissance de 10 suivant leur emplacement dans le nombre.\\
		Par exemple, pour \textcolor{red}{1815} :\\
		\begin{tabular}{p{0.4cm}|p{0.4cm}|p{0.4cm}|p{0.4cm}c}
			$\scriptstyle{10^3}$ & $\scriptstyle{10^2}$ & $\scriptstyle{10^1}$ & $\scriptstyle{10^0}$ &                                                                             \\
			\cline{1-4}
			\textcolor{red}{1}   & \textcolor{red}{8}   & \textcolor{red}{1}   & \textcolor{red}{5}   & \onslide<2->{$=1 \times 1000 + 8 \times 100 + 1\times 10+ 1 \times 1=1815$} \\
		\end{tabular}
	\end{block}
\end{frame}

% Convention d'écriture 
\begin{frame}
	\mframe{\Encodage}
	\begin{block}{Convention d'écriture}
		\begin{itemize}
			\item<1-> Le nombre $101$ pourrait être écris en base 2 (et donc valoir \onslide<2->{cinq)}
			      \onslide<3->{, ou être écrit en base 10, et donc valoir cent un.}
			\item<4-> Afin d'éviter toute confusion, on convient d'écrire le nombre entre parenthèses et de mettre en indice la base dans lequel il est écrit
			\item<5-> Par exemple $\base{10001}{2}$ est le nombre valant, \onslide<6->{dix-sept.}
			\item<6-> Par contre $\base{10000}{10}$ vaut dix mille.
		\end{itemize}
	\end{block}
\end{frame}

% Vocabulaire
\begin{frame}
	\mframe{\Encodage}
	\begin{block}{Vocabulaire}
		\begin{itemize}
			\item<1-> Un chiffre en base 2 s'appelle un \textcolor{red}{bit}, un bit vaut donc 0 ou 1.
			\item<2-> Le regroupement de 8 bits s'appelle un \textcolor{red}{octet}.
			\item<3-> En utilisant un octet, on peut représenter les entiers de 0 à 255.
		\end{itemize}
	\end{block}
\end{frame}


% Exemples
\begin{frame}
	\mframe{\Encodage}
	\begin{exampleblock}{\textcolor{yellow}{\flash} {Question flash}}
		Compléter le tableau de conversion suivant : \\
		\begin{center}
			\begin{tabularx}{0.6\textwidth}{|X|X|}
				\hline
				Ecriture décimale & Ecriture binaire     \\
				\hline
				$\base{142}{10}$  &                      \\
				\hline
				$\base{207}{10}$  &                      \\
				\hline
				                  & $\base{100101}{2}$   \\
				\hline
				$\base{88}{10}$   &                      \\
				\hline
				$\base{222}{10}$  &                      \\
				\hline
				                  & $\base{11100001}{2}$ \\
				\hline
				                  & $\base{11110}{2}$    \\
				\hline
			\end{tabularx}
		\end{center}\end{exampleblock}
\end{frame}


% Exemples
\begin{frame}
	\mframe{\Encodage}
	\begin{exampleblock}{\textcolor{yellow}{\flash} {Question flash}}
		\begin{itemize}
			\item<1-> Ecrire les entiers positifs de 1 à 16 en base 2 :
			      \renewcommand{\arraystretch}{1.5}
			      \begin{tabularx}{0.92\textwidth}{|X|X|X|X|}
				      \hline
				      $(1)_{10}= (\dots)_2$  & $(2)_{10}= (\dots)_2$  & $(3)_{10}= (\dots)_2$  & $(4)_{10}= (\dots)_2$  \\
				      \hline
				      $(5)_{10}= (\dots)_2$  & $(6)_{10}= (\dots)_2$  & $(7)_{10}= (\dots)_2$  & $(8)_{10}= (\dots)_2$  \\
				      \hline
				      $(9)_{10}= (\dots)_2$  & $(10)_{10}=(\dots)_2$  & $(11)_{10}=(\dots)_2$  & $(12)_{10}=(\dots)_2$  \\
				      \hline
				      $(13)_{10}= (\dots)_2$ & $(14)_{10}= (\dots)_2$ & $(15)_{10}= (\dots)_2$ & $(16)_{10}= (\dots)_2$ \\
				      \hline
			      \end{tabularx}
			\item<2->{Combien faudra-t-il de chiffres en base 2 pour écrire $32$ ? \\
			      \lpo[1]}

		\end{itemize}
	\end{exampleblock}
\end{frame}

% Autre base
\begin{frame}
	\mframe{\Encodage}
	\begin{block}{Autre base}
		\begin{itemize}
			\item<1-> Nous savons écrire les entiers naturels en base \textcolor{red}{10} en utilisant \textcolor{red}{10} chiffres, chaque chiffre étant multiplié par une puissance de \textcolor{red}{10}.
			\item<2-> Nous savons écrire les entiers naturels en base \textcolor{red}{2} en utilisant \textcolor{red}{2} chiffres,chaque chiffre étant multiplié par une puissance de \textcolor{red}{2}.
			\item<3-> On montre qu'il est en fait possible, pour tout entier \textcolor{red}{$b \geq 2$} d'écrire les entiers naturels dans la base \textcolor{red}{$b$} en utilisant \textcolor{red}{$b$} chiffres. Chaque chiffre sera alors multiplié par une puissance de \textcolor{red}{$b$}.
		\end{itemize}
	\end{block}
	\begin{exampleblock}{Un exemple en base 5}
		\onslide<4->{
			\begin{tabular}{lcl}
				$\base{421}{5}$ & $=$ & $4 \times 5^2 + 2 \times 5^1 + 1 \times 5^0$ \\
				$\base{421}{5}$ & $=$ & $\base{111}{10}$                             \\
			\end{tabular}\\}
		\onslide<6->{Attention, les chiffres en base 5 sont 0, 1, 2, 3 et 4. Par conséquent écrire $\base{67}{5}$ n'a pas de sens !}
	\end{exampleblock}
\end{frame}


% Ecriture hexadécimale
\begin{frame}
	\mframe{\Encodage}
	\begin{alertblock}{La base 16 : écriture hexadécimale}
		\begin{itemize}
			\item<1-> En informatique, outre la base 2, on utilise aussi beaucoup la base 16.
			\item<2-> En base 16, il y a 16 chiffres : $0,1,2,3,4,5,6,7,8,9$ et $A,B,C,D,E,F$ (n'ayant plus de \og chiffres habituels \fg, on a utilisé les lettres de l'alphabet comme chiffres manquants)
			\item<3-> Comme 16 est une puissance de 2 ($16=2^4$),  on peut aisément passer de l'écriture binaire à l'écriture hexadécimale en regroupant les chiffres en base 2 par groupe de 4. :\\
		\end{itemize}
	\end{alertblock}
\end{frame}


\begin{frame}
	\mframe{\Encodage}
	\begin{block}{Conversion}
		\renewcommand{\arraystretch}{0.8}
		\begin{center}
			\begin{tabularx}{0.4\textwidth}{|X|X|X|}
				\hline
				hex. & bin.  & dec.  \\
				\hline
				0    & 0000  & 0     \\
				1    & 0001  & 1     \\
				2    & 0010  & 2     \\
				3    & 0011  & 3     \\
				4    & 0100  & 4     \\
				5    & 0101  & 5     \\
				6    & \dots & \dots \\
				7    & \dots & \dots \\
				8    & \dots & \dots \\
				9    & \dots & \dots \\
				A    & \dots & \dots \\
				B    & \dots & \dots \\
				C    & \dots & \dots \\
				D    & \dots & \dots \\
				E    & \dots & \dots \\
				F    & \dots & \dots \\
				\hline
			\end{tabularx}
		\end{center}
	\end{block}
\end{frame}

\begin{frame}
	\mframe{\Encodage}
	\begin{block}{Conversion}
		\renewcommand{\arraystretch}{0.8}
		\begin{center}
			\begin{tabularx}{0.4\textwidth}{|X|X|X|}
				\hline
				hex. & bin. & dec. \\
				\hline
				0    & 0000 & 0    \\
				1    & 0001 & 1    \\
				2    & 0010 & 2    \\
				3    & 0011 & 3    \\
				4    & 0100 & 4    \\
				5    & 0101 & 5    \\
				6    & 0110 & 6    \\
				7    & 0111 & 7    \\
				8    & 1000 & 8    \\
				9    & 1001 & 9    \\
				A    & 1010 & 10   \\
				B    & 1011 & 11   \\
				C    & 1100 & 12   \\
				D    & 1101 & 13   \\
				E    & 1110 & 14   \\
				F    & 1111 & 15   \\
				\hline
			\end{tabularx}
		\end{center}
	\end{block}
\end{frame}

%Question Flash
\begin{frame}
	\mframe{\Encodage}
	\begin{exampleblock}{\textcolor{yellow}{\flash} {Question flash}}
		\begin{itemize}
			\item<1-> Ecrire $\base{3EA}{16}$ en base 10
			\item<2-> Ecrire $\base{3EA}{16}$ en base 2
			\item<3-> Ecrire $\base{1101001011}{2}$ en base 16
			\item<4-> Ecrire $\base{1101001011}{2}$ en base 10
		\end{itemize}
	\end{exampleblock}
\end{frame}

%Algorithme des divisions successives
\begin{frame}
	\mframe{\Encodage}
	\begin{block}{\textcolor{yellow}{\flash} {Algorithme des divisions successives}}
		\begin{itemize}
			\onslide<1->{\item L'algorithme des \textcolor{blue}{divisions successives}, permet d'écrire un nombre donnée en base 10 dans n'importe quelle base $b$. Le principe est d'effectuer les divisions euclidiennes successives par $b$, les restes de ces divisions sont les chiffres du nombre dans la base $b$.}
			      \onslide<2->{\item Pour écrire $N$ en base $b$ :}
			      \begin{enumerate}
				      \item<3-> Faire la division euclidienne de $N$ par $b$, soit $Q$ le quotient et $R$ le reste. \\
				            (c'est à dire écrire $N = Q\times b + R$ avec $R<b$)
				      \item<4-> Ajouter $R$ aux chiffres de $N$ en base $b$
				      \item<5-> Si $Q=0$ s'arrêter, sinon recommencer à partir de l'étape 1 en remplaçant $N$ par $Q$.
			      \end{enumerate}
		\end{itemize}
	\end{block}
\end{frame}

%Exemple algorithme des divisions successives
\begin{frame}
	\mframe{\Encodage}
	\begin{exampleblock}{Exemple d'utilisation de l'algorithme des divisions successives}
		Donner l'écriture en base 16 de $\base{2019}{10}$. \\ \pause
		\begin{tabular}{lllllll}
			$2019$                                & $=$               & $\onslide<3->{\textcolor{blue}{126}}$ & $\times$               & $16$               & $+$               & $\onslide<4->{\textcolor{red}{\boxed{3}}} $  \\
			$\onslide<5->{\textcolor{blue}{126}}$ & \onslide<5->{$=$} & $\onslide<6->{\textcolor{blue}{7}}$   & \onslide<5->{$\times$} & \onslide<5->{$16$} & \onslide<5->{$+$} & $\onslide<7->{\textcolor{red}{\boxed{14}}} $ \\
			$\onslide<8->{\textcolor{blue}{7}}$   & \onslide<8->{$=$} & $\onslide<9->{\textcolor{blue}{0}}$   & \onslide<8->{$\times$} & \onslide<8->{$16$} & \onslide<8->{$+$} & $\onslide<10->{\textcolor{red}{\boxed{7}}} $ \\
		\end{tabular} \\
		\onslide<14->{
			Le quotient est nul, l'algorithme s'arrête et les chiffres en base 16 sont les restes obtenus à chaque étape donc  $\base{2019}{10}=\base{7E3}{16}$ (car 14 correspond au chiffre E).}
	\end{exampleblock}
\end{frame}

%Exemple algorithme des divisions successives
\begin{frame}
	\mframe{\Encodage}
	\begin{exampleblock}{Exemple d'utilisation de l'algorithme des divisions successives}
		Donner l'écriture en base 16 de $\base{9787}{10}$. \\ \pause
		\begin{tabular}{lllllll}
			$9787$                                & $=$                 & $\onslide<3->{\textcolor{blue}{611}}$ & $\times$                & $16$                & $+$                & $\onslide<4->{\textcolor{red}{\boxed{11}}} $ \\
			$\onslide<5->{\textcolor{blue}{611}}$ & \onslide<5->{$=$}   & $\onslide<6->{\textcolor{blue}{38}}$  & \onslide<5->{$\times$}  & \onslide<5->{$16$}  & \onslide<5->{$+$}  & $\onslide<7->{\textcolor{red}{\boxed{3}}} $  \\
			$\onslide<8->{\textcolor{blue}{38}}$  & \onslide<8->{$=$}   & $\onslide<9->{\textcolor{blue}{2}}$   & \onslide<8->{$\times$}  & \onslide<8->{$16$}  & \onslide<8->{$+$}  & $\onslide<10->{\textcolor{red}{\boxed{6}}} $ \\
			$\onslide<11->{\textcolor{blue}{2}}$  & \onslide<11->{ $=$} & $\onslide<12->{\textcolor{blue}{0}}$  & \onslide<11->{$\times$} & \onslide<11->{$16$} & \onslide<11->{$+$} & $\onslide<13->{\textcolor{red}{\boxed{2}}} $ \\
		\end{tabular} \\
		\onslide<14->{
			Le quotient est nul, l'algorithme s'arrête et les chiffres en base 16 sont les restes obtenus à chaque étape donc  $\base{9781}{10}=\base{263B}{16}$ (car 11 correspond au chiffre B).}
	\end{exampleblock}
\end{frame}


%Exemple algorithme des divisions successives (base 2)
\begin{frame}
	\mframe{\Encodage}
	\begin{exampleblock}{Exemple d'utilisation de l'algorithme des divisions successives}
		Donner l'écriture en base 2 de $\base{786}{10}$. \\ \pause
		\begin{tabular}{lllllll}
			$786$                                 & $=$                 & $\onslide<3->{\textcolor{blue}{393}}$ & $\times$                & $2$                & $+$                & $\onslide<4->{\textcolor{red}{\boxed{0}}} $  \\
			$\onslide<5->{\textcolor{blue}{393}}$ & \onslide<5->{$=$}   & $\onslide<6->{\textcolor{blue}{196}}$ & \onslide<5->{$\times$}  & \onslide<5->{$2$}  & \onslide<5->{$+$}  & $\onslide<7->{\textcolor{red}{\boxed{1}}} $  \\
			$\onslide<8->{\textcolor{blue}{196}}$ & \onslide<8->{$=$}   & $\onslide<9->{\textcolor{blue}{98}}$  & \onslide<8->{$\times$}  & \onslide<8->{$2$}  & \onslide<8->{$+$}  & $\onslide<10->{\textcolor{red}{\boxed{0}}} $ \\
			$\onslide<11->{\textcolor{blue}{98}}$ & \onslide<11->{ $=$} & $\onslide<12->{\textcolor{blue}{49}}$ & \onslide<11->{$\times$} & \onslide<11->{$2$} & \onslide<11->{$+$} & $\onslide<13->{\textcolor{red}{\boxed{0}}} $ \\
			$\onslide<14->{\textcolor{blue}{49}}$ & \onslide<14->{ $=$} & $\onslide<15->{\textcolor{blue}{24}}$ & \onslide<14->{$\times$} & \onslide<14->{$2$} & \onslide<14->{$+$} & $\onslide<16->{\textcolor{red}{\boxed{1}}} $ \\
			$\onslide<17->{\textcolor{blue}{24}}$ & \onslide<17->{ $=$} & $\onslide<18->{\textcolor{blue}{12}}$ & \onslide<17->{$\times$} & \onslide<17->{$2$} & \onslide<17->{$+$} & $\onslide<19->{\textcolor{red}{\boxed{0}}} $ \\
			$\onslide<20->{\textcolor{blue}{12}}$ & \onslide<20->{ $=$} & $\onslide<21->{\textcolor{blue}{6}}$  & \onslide<20->{$\times$} & \onslide<20->{$2$} & \onslide<20->{$+$} & $\onslide<22->{\textcolor{red}{\boxed{0}}} $ \\
			$\onslide<23->{\textcolor{blue}{6}}$  & \onslide<23->{ $=$} & $\onslide<24->{\textcolor{blue}{3}}$  & \onslide<23->{$\times$} & \onslide<23->{$2$} & \onslide<23->{$+$} & $\onslide<25->{\textcolor{red}{\boxed{0}}} $ \\
			$\onslide<26->{\textcolor{blue}{3}}$  & \onslide<26->{ $=$} & $\onslide<27->{\textcolor{blue}{1}}$  & \onslide<26->{$\times$} & \onslide<26->{$2$} & \onslide<26->{$+$} & $\onslide<28->{\textcolor{red}{\boxed{1}}} $ \\
			$\onslide<29->{\textcolor{blue}{1}}$  & \onslide<29->{ $=$} & $\onslide<30->{\textcolor{blue}{0}}$  & \onslide<29->{$\times$} & \onslide<29->{$2$} & \onslide<29->{$+$} & $\onslide<31->{\textcolor{red}{\boxed{1}}} $ \\
		\end{tabular} \\
		\onslide<32->{
			Le quotient est nul, l'algorithme s'arrête et $\base{786}{10}=\base{1100010010}{2}$.}
	\end{exampleblock}
\end{frame}

\end{document}