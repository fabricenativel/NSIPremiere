\newcommand{\PD}{\cnum Les dictionnaires de Python}

\pythonmode


% Définition des dictionnaires
\begin{frame}
\mframe{\PD}
\begin{alertblock}{Les dictionnaires de Python}
\begin{itemize}
\item<1-> Les dictionnaires de Python permettent de stocker des données sous forme de tableau associant une clé à une valeur : \vspace{0.2cm} \\
\begin{tabularx}{0.8\textwidth}{l|Y|Y|Y|Y|Y|}
 \cline{2-6}
 Valeurs & {\tt val1} & {\tt val2} & {\tt val3} & {\tt val4} & {\tt Val5}   \\
 \cline{2-6}
    \multicolumn{1}{c}{$\uparrow$}      & \multicolumn{1}{c}{$\uparrow$} & \multicolumn{1}{c}{$\uparrow$}  & \multicolumn{1}{c}{$\uparrow$}  & \multicolumn{1}{c}{$\uparrow$}  & \multicolumn{1}{c}{$\uparrow$}   \\
  \multicolumn{1}{c}{Clés} & \multicolumn{1}{c}{\tt 'cle1'} & \multicolumn{1}{c}{\tt 'cle2'} &  \multicolumn{1}{c}{\tt 'cle3'} &  \multicolumn{1}{c}{\tt 'cle4'} &  \multicolumn{1}{c}{\tt 'cle5'} \\        
\end{tabularx}
\item<2-> Un dictionnaire se note entre accolades : \textbf{\{} et \textbf{\}}
\item<3-> Les paires clés/valeurs sont séparés par des virgules.
\item<4-> Le caractère "\textcolor{blue}{\tt :}" sépare une clé de la valeur associée.
\end{itemize}
\end{alertblock} 
\begin{exampleblock}{Exemples}
\begin{itemize}
\item<5-> Un dictionnaire contenant des objets et leurs prix :\\
\onslide<6-> {\tt prix = \{ "verre":12 , "tasse" : 8, "assiette" : 16\} }
\item<7-> Un dictionnaire traduisant des couleurs du français vers l'anglais \\
\onslide<8-> {\tt couleurs = \{ "vert":"green" , "bleu" : "blue", "rouge" : "red" \} }
\end{itemize}
\end{exampleblock}
\end{frame}

% Opérations sur un dictionnaire
\begin{frame}
\mframe{\PD}
\begin{alertblock}{Opérations sur un dictionnaire}
\begin{itemize}
\item<1-> On accède aux éléments d'un dictionnaire avec la syntaxe \textcolor{blue}{\tt nom\_dictionnaire[cle]}\\
\onslide<2->\textcolor{gray}{{\tt prix = \{ "verre":12 , "tasse" : 8, "assiette" : 16, "plat" : 30 \} } \\
Par exemple, {\tt prix["verre"]} contient 12}
\item<3-> On peut ajouter une clé à un dictionnaire existant en effectuant une affectation \textcolor{blue}{\tt nom\_dictionnaire[nouvelle\_cle]=nouvelle\_valeur} \\
\onslide<4->\textcolor{gray}{On ajoute un nouvel objet avec son prix : \\
{\tt prix["couteau"]=20}
}
\item<5-> On peut modifier la valeur associée à une clé avec une affectation \textcolor{blue}{\tt nom\_dictionnaire[cle]=nouvelle\_valeur}\\
\onslide<6->\textcolor{gray}{Le pris d'une tasse passe à 10 : \\
{\tt prix["tasse"]=10}
}
\end{itemize}
\end{alertblock}
\end{frame}

\begin{frame}
\mframe{\PD}
\begin{block}{Présence dans un dictionnaire}
\begin{itemize}
\item<1-> Attention, essayer d'accéder à une clé qui n'est pas dans un dictionnaire renvoie une erreur !\\
\onslide<2-> \textcolor{gray}{Il n'y a pas de clé {\tt 'fourchette'} dans le dictionnaire prix, donc \textcolor{blue}{\tt prix['fourchette']} renvoie une erreur ({\tt \textcolor{red}{KeyError}}).}
\item<3-> On teste la présence d'une clé dans un dictionnaire avec \textcolor{blue}{\tt cle in nom\_dictionnaire}\\
\onslide<4->\textcolor{gray}{la fourchette n'est pas dans le dictionnaire prix \\
Le test \textcolor{blue}{\tt fourchette in prix} renvoie \textcolor{blue}{\tt False}}
\item<5-> On peut supprimer une clé existante dans un dictionnaire avec \textcolor{blue}{\tt del nom\_dictionnaire[cle]}\\
\onslide<6->\textcolor{gray}{On supprimer le couteau : \\
\textcolor{blue}{\tt del prix["couteau"]}
}
\end{itemize}
\end{block}
\end{frame}


% Opérations sur un dictionnaire
\begin{frame}
\mframe{\PD}
\begin{alertblock}{Parcours d'un dictionnaire}
\begin{itemize}
\item<1-> Le parcours par clé s'effectue directement avec \textcolor{blue}{\tt for cle in nom\_dictionnaire}\\
\onslide<2->\textcolor{gray}{{\tt prix = \{ "verre":12 , "tasse" : 8, "assiette" : 16, "plat" : 30 \} } \\
Par exemple, {\tt for objet in prix} permettra à la variable {\tt objet} de prendre successivement les valeurs des clés : {\tt "verre", "tasse", "assiette"} et {\tt "plat"}.}
\item<3-> Le parcours par valeur s'effectue en ajoutant \textcolor{blue}{\tt .values()} au nom du dictionnaire : \textcolor{blue}{\tt for valeur in nom\_dictionnaire.values() \\} 
\onslide<4->\textcolor{gray}{
Par exemple, {\tt for p in prix.values()} permettra à la variable {\tt p} de prendre successivement les valeurs du dictionnaire : {\tt 12, 8 , 16} et {\tt 30}.
}
\end{itemize}
\end{alertblock}
\end{frame}



