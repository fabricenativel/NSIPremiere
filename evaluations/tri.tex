\documentclass[11pt,a4paper]{article}

\usepackage{ActCor}
\usepackage{listings}

\begin{document}
\input{\detokenize{/home/fenarius/Travail/Cours/Commun/latex/Macros.tex}}

\DevoirNSI{Algorithme de tri}{\Pre}\vspace{0.2cm}
\pythonmode
%Nom de la première activité

\Exo{Algorithmes de tri}{}
\QListe
\item Tri par sélection
\SQListe
\item Expliquer en quelques phrases le principe de l'algorithme du tri par sélection
\item On applique l'algorithme du tri par sélection à la liste  {\tt [11,17,14,42,5,30,19]}. Donner l'évolution du contenu de la liste au cours des étapes du tri.
\FinListe
\item Tri par insertion
\SQListe
\item Expliquer en quelques phrases le principe de l'algorithme du tri par insertion
\item On applique l'algorithme du tri par insertion à la liste  {\tt [11,17,14,42,5,30,19]}. Donner l'évolution du contenu de la liste au cours des étapes du tri.\\
\aide \; On pourra donner l'état de la liste après chaque insertion, sans écrire les étapes intermédiaires de ces insertions.
\FinListe
\FinListe

\vspace{0.2cm}
\Exo{Programmation en python}{}\\
On veut écrire une fonction {\tt trouve\_mini} qui renvoie l'indice de la première occurrence du plus petit élément d'une liste. Par exemple, {\tt trouve\_mini([13,9,10,7,18])} doit renvoyer 3.
\QListe
\item Que doit renvoyer {\tt trouve\_mini([15,10,17,11,22])} ?
\item Que doit renvoyer {\tt trouve\_mini([5,22,5,41,20,5,17])} ? \\
\aide \; On rappelle que lorsque le minimum apparaît plusieurs fois, on doit renvoyer la première occurence.
\item Compléter le code de la fonction {\tt trouve\_mini} ci-dessous.
\begin{lstlisting}
... trouve_mini(liste):
indice_mini = 0
for indice in range(.......):
    if .......<........:
        indice_mini = indice
...... indice_mini
\end{lstlisting}
\item Que renverra {\tt trouve\_mini([])} ?
\item Modifier cette fonction afin que {\tt trouve\_mini([])} renvoie $-1$.
\FinListe

\vspace{0.2cm}

\Exo{Coût}{}
\QListe
\item Expliquer rapidement ce que signifie une complexité linéaire pour un algorithme.
\item Donner au moins un exemple, d'algorithme ayant une compléxité linéaire.
\item Un algorithme de complexité linéaire traite une liste de $25\,000$ éléments en $0,03s$. Quel est le temps approximatif d'exécution prévisible pour une liste de $150\,000$ éléments ?
\item On rappelle que: \og l'algorithme du tri par sélection à une \textit{complexité quadratique} \fg, expliquer en quelques lignes ce que signifie cette phrase.
\item On a écrit un programme permettant de trier une liste à l'aide du tri par sélection. Si ce programme traite une liste de $5\,000$ élément en 0,75 secondes, quel est le temps approximatif d'exécution prévisible pour une liste de $80\,000$ éléments ?
\FinListe

\end{document}

