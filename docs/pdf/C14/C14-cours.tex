\PassOptionsToPackage{dvipsnames,table}{xcolor}
\documentclass[10pt]{beamer}
\usepackage{Cours}

\begin{document}

\input{\detokenize{/home/fenarius/Travail/Cours/NSIPremiere/docs/commun/MacrosCours.tex}}
\setcounter{numchap}{14}

\pythonmode

\newcommand{\FT}{\cnum Fusion de tables}

\pythonmode

% Opérateur de concaténation
\begin{frame}
    \mframe{\FT}
    \begin{block}{Opérateur de concaténation}
        Lorsque deux tables possèdent exactement les même descripteurs, on peut les fusionner en construisant leur réunion. On dit que l'on effectue la \textcolor{blue}{concaténation} des deux tables et l'opérateur correspodant en Python est \textcolor{blue}{+}.
    \end{block}
    \begin{exampleblock}{Exemple}
        On dispose de deux tables \tt{notes\_1nsi1} (contenant les notes du groupe 1NSI1) à un devoir et \tt{notes\_1nsi2} contenant les notes du groupe 1NSI2 à ce même devoir :
        {\footnotesize
        \tt{notes\_1nsi1=[{"Prénom":"Jacques","Note":12},{"Prénom":"Gilbert","Note":7},....]}
        \tt{notes\_1nsi2=[{"Prénom":"Mélanie","Note":18},{"Prénom":"Sophie","Note":17},....]}
        }\\
        Les descripteurs sont identiques (\tt{"Prénom"} et \tt{"Note"}), on peut donc regrouper les deux tables en effectuant : \\
        \tt{notes\_1nsi = notes\_1nsi1} \pmc{+} \tt{notes\_1nsi2}
    \end{exampleblock}
\end{frame}

% Remarque
\begin{frame}
    \mframe{\FT}
    \begin{block}{Remarques}
        \begin{itemize}
            \item<1-> La concaténation peut poser problème lorsque les enregistrements ne sont pas identifiés de façon  unique. Dans l'exemple ci-dessous cela correspond au cas où deux élèves de groupe différents auraient le même prénom.
            \item<2-> Deux tables peuvent avoir les mêmes descripteurs mais des \textcolor{blue}{domaine de validité} différents. Dans l'exemple précédent, si les notes du groupe 1 sont sur 20, leur domaine de validité est l'intervalle $[0,20]$.
            \item<3-> On fera donc attention lors de la concaténation à vérifier que :
            \begin{itemize}
                \item <4-> Les enregistrements sont identifiés de façon unique
                \item <5-> Les données sont homogènes (exprimées dans la même unité, et avec le même domaine de validité)
            \end{itemize}
        \end{itemize}
        \onslide<6->{Ces aspects seront traités de façon plus approfondie en terminale avec les bases de données.}
    \end{block}
\end{frame}


% Schéma concaténation
\begin{frame}
    \mframe{\FT}
    \begin{block}{Schématisation de la concaténation}
        \begin{center}
        \begin{tabular}{|c|c|c|}
            \hline
            \cellcolor{green!25}{\quad} & \cellcolor{green!25}{\quad} & \cellcolor{green!25}{\quad} \\
            \hline
             ... & ... & ... \\
            \hline
            ... & ... & ... \\
            \hline
            ... & ... & ... \\
            \hline
        \end{tabular} {\huge\textcolor{red}{+}} 
        \begin{tabular}{|c|c|c|}
            \hline
            \cellcolor{green!25}{\quad} & \cellcolor{green!25}{\quad} & \cellcolor{green!25}{\quad} \\
            \hline
             ... & ... & ... \\
            \hline
            ... & ... & ... \\
            \hline
        \end{tabular} {\huge\textcolor{red}{$\rightarrow$}} 
        \begin{tabular}{|c|c|c|}
            \hline
            \cellcolor{green!25}{\quad} & \cellcolor{green!25}{\quad} & \cellcolor{green!25}{\quad} \\
            \hline
             ... & ... & ... \\
            \hline
            ... & ... & ... \\
            \hline
            ... & ... & ... \\
            \hline
            ... & ... & ... \\
            \hline
            ... & ... & ... \\
            \hline
        \end{tabular}
    \end{center}
    Les descripteurs des deux tables sont identiques.
    \end{block}
\end{frame}

% Opérateur de concaténation
\begin{frame}
    \mframe{\FT}
    \begin{block}{Jointure de deux tables}
        Lorsque deux tables ont au moins un descripteur commun, on peut les fusionner en créant leur \textcolor{red}{jointure}. 
        On rapproche les enregistrements de chacune des deux tables lorsque le champ commun coïncide.
    \end{block}
    \begin{exampleblock}{Exemple}
        On dispose de deux tables \tt{notes\_d1} (contenant les notes d'une classe à un devoir) et \tt{notes\_d2} contenant les notes de cette même classe à un autre devoir :
        {\footnotesize
        \tt{notes\_d1=[{"Prénom":"Jacques","d1":12},{"Prénom":"Gilbert","d1":17},....]}
        \tt{notes\_d2=[{"Prénom":"Jacques","d2":16},{"Prénom":"Gilbert","d2":13},....]}
        }\\
        Les deux tables ont en commun le descripteur {\tt Prénom} la jointure donne : \\
        {\footnotesize \tt{[{"Prénom":"Jacques","d1":12,"d2":16},{"Prénom":"Gilbert","d1":17,"d2":13},....]}}
    \end{exampleblock}
\end{frame}

% Remarque
\begin{frame}
    \mframe{\FT}
    \begin{block}{Remarques}
        \begin{itemize}
            \item<1-> Le (ou les) descripteur(s) commun(s) doit (doivent) permettre d'identifier de façon unique chaque enregistrement. Dans l'exemple ci dessous, deux élèves de la classe ne doivent pas avoir le même prénom. 
            \item<2-> La jointure est une opération fondamentale des bases de données et sera étudiée en terminale.
        \end{itemize}
    \end{block}
\end{frame}


% Schéma concaténation
\begin{frame}
    \mframe{\FT}
    \begin{block}{Schématisation de la jointure}
        \begin{center}
        \begin{tabular}{|c|c|c|}
            \hline
            \cellcolor{green!25}{\quad} & \cellcolor{red!25}{\quad} & \cellcolor{red!25}{\quad} \\
            \hline
             ... & ... & ... \\
            \hline
            ... & ... & ... \\
            \hline
        \end{tabular} {\huge\textcolor{red}{+}} 
        \begin{tabular}{|c|c|c|}
            \hline
            \cellcolor{green!25}{\quad} & \cellcolor{blue!25}{\quad} & \cellcolor{blue!25}{\quad} \\
            \hline
             ... & ... & ... \\
            \hline
            ... & ... & ... \\
            \hline
        \end{tabular} {\huge\textcolor{red}{$\rightarrow$}} 
        \begin{tabular}{|c|c|c|c|c|}
            \hline
            \cellcolor{green!25}{\quad} & \cellcolor{red!25}{\quad} & \cellcolor{red!25}{\quad} & \cellcolor{blue!25}{\quad} & \cellcolor{blue!25}{\quad} \\
            \hline
             ... & ... & ... &  ... & ... \\
            \hline
            ... & ... & ...  & ... & ... \\
            \hline
        \end{tabular}
    \end{center}
    Les deux tables ont (au moins) un descripteur commun.
    \end{block}
\end{frame}

% Remarque
\begin{frame}
    \mframe{\FT}
    \begin{alertblock}{Synthèse}
        La \textcolor{red}{fusion} de deux tables de données peut s'effectuer :
        \begin{itemize}
            \item<1-> par \textcolor{red}{concaténation} lorsque les deux tables possèdent des descripteurs identiques.
            \item<2-> par \textcolor{red}{jointure} lorsque les deux tables possèdent des colonnes communes. On met alors en relation les enregistrements qui coïncide sur les champs communs.
        \end{itemize}
    \end{alertblock}
\end{frame}


\end{document}



