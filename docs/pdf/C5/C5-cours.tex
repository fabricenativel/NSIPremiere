\PassOptionsToPackage{dvipsnames,table}{xcolor}
\documentclass[10pt]{beamer}
\usepackage{Cours}

\begin{document}

\input{\detokenize{/home/fenarius/Travail/Cours/NSIPremiere/docs/commun/MacrosCours.tex}}
\setcounter{numchap}{5}
\newcommand{\Python}{\cnum Initiation à Python avec turtle}

\pythonmode

% Instructions conditionnelles
\begin{frame}[fragile]
	\mframe{\Python}
	\begin{alertblock}{Instructions conditionnelles}
		\begin{itemize}
			\item<2-> La syntaxe d'une instruction conditionnelle en Python est :
			      \begin{lstlisting}
	if <condition>:
		<instructions1>
	else:
		<instructions2>
	\end{lstlisting}
			      Cela permet d'exécuter les {\tt <instructions1>} si la {\tt condition} est vérifiée, sinon on exécute les {\tt <instructions2>}.
			\item<3->  \textcolor{red}{\danger} On fera bien attention à la syntaxe du langage, et notamment à l'usage du caractère \textcolor{red}{\tt :} qui suit la condition (et le {\tt else}) et à l'\textcolor{red}{indentation}, c'est à dire le décalage des instructions qui doivent s'executer.
		\end{itemize}
	\end{alertblock}
\end{frame}

% Exemple instructions conditionnelles
\begin{frame}[fragile]
	\mframe{\Python}
	\begin{exampleblock}{Exemples}
		\begin{enumerate}
			\item<1-> Ecrire l'instruction permettant de tester si la variable {\tt erreurs} vaut 0
			\item<2-> On suppose qu'une variable {\tt longueur} peut être positive ou négative, si cette variable est positive alors on fait avancer la tortue de {\tt longueur}, sinon on la fait reculer de {\tt -longueur}.
			      Ecrire les instructions python correspondantes.
		\end{enumerate}
	\end{exampleblock}
\end{frame}

% Exemple instructions conditionnelles
\begin{frame}[fragile]
	\mframe{\Python}
	\begin{exampleblock}{Exemples}
		\begin{enumerate}
			\item Ecrire l'instruction permettant de tester si la variable {\tt erreurs} vaut 0
			      \begin{lstlisting}
    if erreur==0:
		\end{lstlisting}
			\item On suppose qu'une variable {\tt longueur} peut être positive ou négative, si cette variable est positive alors on fait avancer la tortue de {\tt longueur}, sinon on la fait reculer de {\tt -longueur}.
			      Ecrire les instructions python correspondantes.
			      \begin{lstlisting}
		if longueur>0:
			crayon.forward(longueur)
		else:
			crayon.backward(-longueur)
	\end{lstlisting}
		\end{enumerate}
	\end{exampleblock}
\end{frame}


% boucle while
\begin{frame}[fragile]
	\mframe{\Python}
	\begin{alertblock}{Boucles {\tt while}}
		\begin{itemize}
			\item<2-> La syntaxe d'une boucle \textcolor{red}{\tt while}  en Python est :
			      \begin{lstlisting}
	while <condition>:
		<instruction>
	\end{lstlisting}
			      Cela permet d'exécuter les {\tt <instructions>} tant que la {\tt <condition>} est  vérifiée.
			\item<3->  On ne sait pas a priori combien de fois cette boucle sera exécutée (et elle peut même être infinie), on dit que c'est une boucle \textcolor{blue}{non bornée}.
		\end{itemize}
	\end{alertblock}
\end{frame}

% boucle while
\begin{frame}[fragile]
	\mframe{\Python}
	\begin{exampleblock}{Exemple d'une boucle {\tt while}}
		On suppose déjà crée une fonction {\tt carre(c)} qui dessine un carré de côté {\tt c} à partir de la position courante de la tortue.
		Ecrire un programme Python, permettant de tracer la figure suivante sachant que : 
		\begin{itemize}
		\item le carré initial à 200 pixels de côté
		\item le côté des carrés intérieur diminue de dix pourcents à chaque étape
		\item le plus petit carré a un côté mesurant plus de 5 pixels.
		\end{itemize}
		\begin{center}
			\includegraphics[scale=0.3]{ex_cours.eps}
		\end{center}
	\end{exampleblock}
\end{frame}


% boucle while
\begin{frame}[fragile]
	\mframe{\Python}
	\begin{exampleblock}{Exemple d'une boucle {\tt while}}
		\begin{lstlisting}
 	cote = 200
	while cote>5:
    	carre(cote)
     	cote = cote * 0.9
		\end{lstlisting}
		\begin{center}
			\includegraphics[scale=0.3]{ex_cours.eps}
		\end{center}
	\end{exampleblock}
\end{frame}

% boucle while
\begin{frame}[fragile]
	\mframe{\Python}
	\begin{alertblock}{Fonction renvoyant un résultat}
		En plus d'exécuter un bloc d'instructions, une fonction peut transmettre une valeur au reste du programme à l'aide d'une instruction \pmc{return}.
		On utilise alors la syntaxe suivante : 
		\begin{lstlisting}
			def <nom_fonction>(<arguments>):
				<instruction>
				return <valeur>
			\end{lstlisting}
	\end{alertblock}
\end{frame}

% Exemple
\begin{frame}[fragile]
	\mframe{\Python}
	\begin{exampleblock}{Exemple de fonction contenant un \pmc{return}}
		La fonction ci-dessous, renvoie la moyenne des deux nombres donnés en argument
		\begin{lstlisting}
 	def moyenne(x,y):
		m = (x+y)/2
		return m
		\end{lstlisting}
	\end{exampleblock}
\end{frame}



\end{document}
